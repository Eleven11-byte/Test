\documentclass[utf8,a4paper]{book} 

\usepackage{geometry}  
% 与页边距有关的宏包
\geometry{
left=2.0cm, 
right=2.0cm, 
top=2.5cm, 
bottom=2.5cm}  
%设置页边距,与Word一致
\usepackage{ragged2e} 
%与两端对齐有关的宏包
\usepackage{color}
% 文本颜色处理 {\color{red}这里是红色}
\usepackage{setspace}
%支持行间距的宏包,用它可以调整行间距
\usepackage{ctex}  
%%支持中文的包
\usepackage[utf8]{inputenc}
%! font encoding
\usepackage[T1]{fontenc}   
%与Times New Roman相关的包
\usepackage{graphicx}     
\usepackage{subfigure} 
%支持插入图片的包
\usepackage{float}       
%支持插入图片的包
\usepackage{underscore}
%支持正文中的下划线

\usepackage{bm}  %
%数学中公式符号黑体表示变量 \bm{}
\usepackage{amsmath}  
%数学中公式符号表示变量 \mathcal{}
\usepackage{mathrsfs}  
%数学中公式符号手写花体 \mathscr{}
\usepackage{amsfonts,amssymb}

\usepackage{color} 
% 文本颜色处理 {\color{red}这里是红色}

\usepackage{listings}
%显示Python代码的包 \begin{lstlisting}
\definecolor{dkgreen}{rgb}{0,0.6,0}
\definecolor{gray}{rgb}{0.5,0.5,0.5}
\definecolor{mauve}{rgb}{0.58,0,0.82}
\lstset{frame=tb,
  language=Python,
  aboveskip=3mm,
  belowskip=3mm,
  showstringspaces=false,
  columns=flexible,
  basicstyle={\small\ttfamily},
  numbers=none,
  numberstyle=\tiny\color{gray},
  %keywordstyle=\color{blue},
  %commentstyle=\color{dkgreen},
  %stringstyle=\color{mauve},
  breaklines=true,
  breakatwhitespace=true,
  tabsize=3
}

\usepackage{hyperref}
%支持链接
\hypersetup{
colorlinks=true,
linkcolor=black
}

\usepackage{multirow}
\usepackage{multicol}
\usepackage{array}
\usepackage{booktabs}
\usepackage{bigstrut}
\usepackage{makecell}
\usepackage{listings}
\usepackage{tabularx}

\usepackage[table]{xcolor}
\definecolor{RowBack}{rgb}{0.9,0.9,0.9}

\usepackage[noend]{algpseudocode}
\usepackage{algorithmicx,algorithm}

\usepackage{pifont}
%带圈数字 \ding{172-181}

\usepackage{caption2}
%处理图片标题名称中的冒号
\usepackage{bbding}

\usepackage{enumerate}

%编译报错中提到需要使用bookmark
\usepackage{bookmark}

\begin{document}   
\setmainfont{Times New Roman}  
\setcounter{tocdepth}{4}
\setcounter{secnumdepth}{4}
\renewcommand\arraystretch{1.2}
\renewcommand{\figurename}{图}
\renewcommand{\captionlabeldelim}{}

\chapter{章节名称}

正文正常写

\section{第一节}

正文正常写

\subsection{插入代码}

代码引用示例如下(去除了颜色)

\begin{lstlisting}
    public static void Bilt(Texture src,RenderTexture dest);
    public static void Bilt(Texture src,RenderTexture dest,Material mat,int pass =-1);
    public static void Bilt(Texture src,Material mat,int pass=-1);
    
        // Called when start
            protected void CheckResources() {
                bool isSupported = CheckSupport();
        
                if (isSupported == false) {
                    NotSupported();
                }
            }
        
        // Called in CheckResources to check support on this platform
            protected bool CheckSupport() {
                if (SystemInfo.supportsImageEffects == false || SystemInfo.supportsRenderTextures == false) {
                    Debug.LogWarning("This platform does not support image effects or render textures.");
                    return false;
                }
        
                return true;
            }


\end{lstlisting}

\subsection{插入图片}

图片插入示例\ref{brtsatcon}所示。
    \begin{figure}[htbp]
        \centering
        \includegraphics[width=.8\linewidth]{./images/brtsatcon.png}%相对路径
        \caption{图像调整效果:左图为原图,右图为调整后图像}
        \label {brtsatcon}
    \end{figure}

\section{第二节}

\end{document}